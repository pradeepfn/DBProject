\section{Implementation}
%how to validate implementation
\noindent{\bf Persistent memory hardware} We emulate persistent memory using volatile DRAM memory. NVMs are slower than DRAM, so we account for 
slow NVM latency using memory throttling. The modern CPUs allow us to configure the duty cycle value in the memory channel. We use that
facility to throttle the bandwidth of the DRAM memory device, reducing the bandwidth and latency of the DRAM reads/writes. 

\noindent{\bf Persistent containers}
We use persistent data-structure implementations of Intel PMDK to model our persistent containers. The PMDK's goal is not to 
create persistent containers. Hence we had to wrap some of their persistent memory data-structures to fit our container model.
e.g.: Wrapping B+tree with map interface.

\noindent{\bf Data Replication}

We use log replication system software layer called cyclone~\cite{cyclone} for our data replication. Cyclone implements 
state machine replication with Raft as the consensus protocol. Cyclone network transports use highly optimized 
userspace Intel's data plane development kit (DPDK) transport stack. Due to this, we had to port YCSB benchmark to
use same DPDK transport for performance reasons.

\section{Evaluation}
%evaluate performance

First we evaluate the basic performance characteristic of our proposed system using well known benchmark -- YCSB. 
We chose YCSB, because its key-value operations fits well for our container interfaces. We then compare and contrast
the performance numbers with state of the art key-value store from Facebook -- RocksDB ~\cite{rocksdb}. Work items involve;
\begin{itemize}
\item Port containers to work with replication software
\item Port RocksDB to work with cyclone.
\item Port YCSB to work with DPDK transport
\item Runs benchmarks, gettingn numbers
\end{itemize}


Next,we evaluate our system with a real application workload. In this experiment setup, we try to understand the costs
and benefits involved while using a reliable persistent containers as the basic application's peristent storage abstraction.
To that end we plan to port a well known application to use our storage primitives. Work items involve;
\begin{itemize}
\item Porting existing app to work with our generalized remote Blizzard API.
\item Running benchmark on the application, with and without Blizzard storage model.
\end{itemize}


\section{Project Goals}
%project goals description
We have identified the following goals for our project.
\begin{itemize}
	\item {\bf The 75$\%$ goal:} 1) Porting a Tree (range lookup) and Map ( point lookup) data structure as a reliable persistent containers. 2) Implementing
			client side library routines to interact with those containers over the network. 

		\item {\bf The 100$\%$ goal:} 1) Port YCSB benchmark to work with our containers. 2) Port a state of the art kv-store (RocksDB). 
				3) Using ported YCSB benchmark client, report numbers for Blizzard containers and competing state of the art. ( RocksDB ).

\item {\bf The 125$\%$ goal:} Extend and PMWCAS to support reliability. This include design of the primive as it is not straight forward at this
		point to design it.
\end{itemize}



\section{Related Work}

PMWCAS~\cite{pmwcas} is a programming primtive that supports both persistency and multiword updates.
It extends well known CAS pritmive with multiword support while supporting same all or nothing
programming semantics of CAS. Furthermore, it extends the CAS to be persistent memory aware by
adding support for durability semantics ( fences and cache-line flushes).

Intel PMDK~\cite{pmdk} provides persistent memory aware data-structures. But they lack the reliability semantics.

